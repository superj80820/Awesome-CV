\documentclass[12pt]{article}

\usepackage{graphicx}
\usepackage{amssymb}
\usepackage[encapsulated]{CJK}    % 打中文用的CJK套件

\begin{document}
\begin{CJK*}{UTF8}{bsmi}
% bsmi是字形, 有bsmi,bkai,gbsn,gkai,前兩個字體較不會缺字


% 更改表格、圖片標題上的Table、Figure文字成中文
\renewcommand{\tablename}{表}
\renewcommand{\figurename}{圖}
\renewcommand{\refname}{參考文獻}

\title{\LaTeX{} 中文輸入環境(CJK-UTF8)範例}
\author{Yung-Hsiang Chiu}
\maketitle

\section{MikTex、WinEdt軟體安裝與設定步驟}















\begin{enumerate}
\item 下載並安裝MikTex。(bin資料夾內附有基本版MikTex安裝檔案:basic--miktex--2.9.4521.exe)
\item 安裝WinEdt並註冊,開啟WinEdt並點擊工具列上的Help$\rightarrow$Register WinEdt輸入註冊資訊。(bin資料夾內附有WinEdt安裝檔:winedt55.exe和安裝資訊檔 wineditreginfo.exe)
\item 啟動WinEdt的MikTex自動設定,點擊工具列上的Options$\rightarrow$Configuration Wizard),並按下圖\ref{fig:MikTexConfigInWinEdt}中紅框表示的兩處按鈕,以完成啟動MikTex的功能。
\item 開啟本tex檔案(ChineseUTF8.tex)並點擊TeXify的圖示編譯成dvi檔案。

由於 Big5 編碼模式(Windows預設的文字模式)會有許功蓋問題,例如:"程式"這兩個字打在tex中就會造成編譯錯誤。
我目前尚未找到解決辦法,所以只能建議各位使用 UTF-8 編碼模式。

使用UTF-8編碼模式的缺點在於WinEdt開啟TeX檔案時,必須選擇開啟的"檔案類型"為:UTF-8,如圖\ref{fig:dia_utf8mode}所示,
若是沒有用UTF-8開啟,則中文部分都會變成亂碼並且無法正確編輯檔案。當我們開啟Tex檔案時,使用了UTF-8的編碼模式,
則會在WinEdt的介面最下方的狀態列上顯示TeX:UTF-8,如圖\ref{fig:utf8mode}中紅框處所示;
反之,使用Big5編碼模式開啟的TeX檔案,只會在狀態列上顯示TeX。

\item 編譯成dvi時,MikTex若發現有缺少了某些套件,會自動搜尋網路上可用的套件,然後徵詢我們的意見,因此只需要將電腦接上網路,
    同意安裝所有缺少的套件,即可完成Miktex中文輸入環境的安裝,約略會需要同意下載4個套件。
\item dvi編譯成功後,按下dvi2pdf圖示,即可將dvi檔案輸出成pdf檔案。
\end{enumerate}



\end{CJK*} % 結束使用CJK提供的打中文功能
\end{document}

最後,若是使用中文時無法成功編譯,有很大的可能是因為Section的名稱、或是Equation、Figure及Table的標題
($\backslash\mbox{caption}\{\dots\}$)的中文字產生編譯器內部處理問題。像是文件的起頭使用a4paper設定就會產生編譯問題
($\backslash \mbox{documentclass}[12\mbox{pt,a4paper}]\{\mbox{article}\}$)。

% 圖片
\begin{figure}
\centering
\includegraphics[scale=0.5]{MikTexConfigInWinEdt.eps}
\caption{啟動WinEdt的MikTex自動設定。}\label{fig:MikTexConfigInWinEdt}
\end{figure}

\begin{figure}
\centering
\includegraphics[scale=0.5]{OpenDialogUTF8.eps}
\caption{開啟UTF-8編碼模式的TeX檔案}\label{fig:dia_utf8mode}
\end{figure}

\begin{figure}
\centering
\includegraphics[scale=0.5]{LatexUTF8Mode.eps}
\caption{正在編輯一個以UTF-8模式開啟的TeX檔案。}\label{fig:utf8mode}
\end{figure}



\section{中文測試小節標題1}

\textbf{這是CJK-UTF8中文輸入測試範例檔案。}

這是中文測試檔案。\\測試手動換行

測試換段,換段和換行的差異在於首行縮排2個字元

測試引用資料\cite{Bayer1976,evl:bdpsnr,klchung}

\section{標題2針對公式, 圖片及表格測試}

%公式
\begin{equation}
    p=\frac{1}{\sqrt{2\pi\sigma^2}}
\end{equation}

% 表格
\begin{table}
\caption{Average improvement ratios平均增進百分比}\label{tbl:4}
\begin{center}
\begin{tabular}{ccccccc}
\hline    & \multicolumn{2}{c}{前人的方法} & \multicolumn{2}{c}{提出的方法} & \multicolumn{2}{c}{提出的方法2}\\
\hline
       QP &
       $\vartriangle$位元率 & $\vartriangle$PSNR &
       $\vartriangle$位元率 & $\vartriangle$PSNR &
       $\vartriangle$位元率 & $\vartriangle$PSNR \\
\hline
       20 &  -3.4\% & 0.5\% & -20.0\% & 0.4\% & -33.9\% & 0.3\% \\
       32 &  -7.9\% & 0.4\% & -27.4\% & 0.8\% & -46.5\% & 3.1\% \\
       42 & -13.3\% & 0.7\% & -31.1\% & 1.7\% & -48.2\% & 7.1\% \\
       44 & -14.0\% & 0.8\% & -30.8\% & 2.0\% & -45.3\% & 7.8\% \\
\hline
       \textbf{Avg.}& \textbf{-7.2\%} & \textbf{0.6\%} & \textbf{-24.2\%} & \textbf{1.0\%} & \textbf{-39.9\%} &  \textbf{2.9\%} \\
\hline
\end{tabular}
\end{center}
\end{table}


\begin{thebibliography}{1} % 參考資料開始

\bibitem{Bayer1976}
B. E. Bayer, ``Color imaging array," U.S. Patent\#$3971065$, 1976.

\bibitem{evl:bdpsnr}
G. Bjontegaard, ``Calculation of average PSNR differences between RD curves," VECG-M33, ITU-T VECG Meeting, Austin, Texas, 2--4, 2001.

\bibitem{klchung}
鍾國亮, 影像處理與電腦視覺 第五版, March 2012.

\end{thebibliography} % 參考資料結束



\end{CJK*} % 結束使用CJK提供的打中文功能
\end{document}
